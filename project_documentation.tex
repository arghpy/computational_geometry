\documentclass{article}
\usepackage{lmodern}
\usepackage[english]{babel}
\usepackage[colorlinks=true, allcolors=blue]{hyperref}

    \title{\textbf{Project Documentation}}
    \author{Andrei Suba, Silaghi-Fartan Stefan\\
	West University of Timișoara,\\
	Timișoara, Romania
    }
    \date{}
    
    \addtolength{\topmargin}{-3cm}
    \addtolength{\textheight}{3cm}

\usepackage{listings}
\usepackage{color}

\definecolor{dkgreen}{rgb}{0,0.6,0}
\definecolor{gray}{rgb}{0.5,0.5,0.5}
\definecolor{mauve}{rgb}{0.58,0,0.82}

\lstset{frame=tb,
  language=C,
  aboveskip=3mm,
  belowskip=3mm,
  showstringspaces=false,
  columns=flexible,
  basicstyle={\small\ttfamily},
  numbers=none,
  numberstyle=\tiny\color{gray},
  keywordstyle=\color{blue},
  commentstyle=\color{dkgreen},
  stringstyle=\color{mauve},
  breaklines=true,
  breakatwhitespace=true,
  tabsize=3
}

\begin{document}

\maketitle
\thispagestyle{empty}

\begin{abstract}

\noindent In this paper, we present a small application written in Python that allows a user to interactively create points in two-dimensional space using their mouse. The application then computes the convex layers of the resulting point set and displays the resulting figure. The goal of this project was to develop a simple and intuitive tool for visualizing and understanding the concept of convex layers in a geometric context. The application is implemented using a combination of Python's built-in graphical user interface (GUI) library, Tkinter, and the Jarvis's Algorithm (or Gift Wrapping Algorithm) for computing the convex hull of points. We discuss the design and implementation of the application, as well as its potential applications in educational settings and beyond.

\end{abstract}

\clearpage

\tableofcontents

\clearpage

\section{Introduction}

Computational geometry is a crucial field in computer science that deals with the efficient representation and manipulation of geometric objects. One important concept in this field is the convex layer, which refers to the smallest convex polygon that encloses a given set of points in two-dimensional space. In this paper, we present a small application that allows users to interactively create points using their mouse and visualize the resulting convex layers in real-time. The application was developed using Python and makes use of the Tkinter library for the graphical user interface and the Jarvis's Algorithm for computing convex hulls. Our goal in creating this application was to provide a simple and intuitive tool for visualizing and understanding convex layers in a geometric context. In the following sections, we will describe the design and implementation of the application in detail and discuss its potential applications and future directions.

\section{Motivation}
A
s part of our university course on computational geometry, we were required to develop a project that demonstrated our understanding of the subject. We decided to focus on convex layers, as we felt that this was an important concept in the field. 
\newline
\newline
Our goal was to develop a simple and intuitive tool that would make the concept of convex layers more accessible and engaging for students and other learners. We believe that our small application has the potential to be a useful resource in educational settings where visualization makes for a better understanding. We hope that our project will provide a valuable learning experience for both ourselves and our users.

\section{Design and Implementation}

\section{Evaluation and Results}

\section{Applications and Future Work}

\section{Conclusion}

\end{document}

